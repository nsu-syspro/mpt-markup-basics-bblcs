% Этот шаблон документа разработан в 2014 году
% Данилом Фёдоровых (danil@fedorovykh.ru) 
% для использования в курсе 
% <<Документы и презентации в \LaTeX>>, записанном НИУ ВШЭ
% для Coursera.org: http://coursera.org/course/latex .
% Вы можете изменять, использовать, распространять
% этот документ любым способом по своему усмотрению. 
% В качестве благодарности автору вы можете сохранить 
% в начале документа данный текст или просто ссылку на
% http://coursera.org/course/latex
% Исходная версия Шаблона --- 
% https://www.writelatex.com/coursera/latex/1.1


\documentclass[a4paper,12pt]{article}

\usepackage{cmap}					% поиск в PDF
\usepackage[T2A]{fontenc}			% кодировка
\usepackage[utf8]{inputenc}			% кодировка исходного текста
\usepackage[english,russian]{babel}	% локализация и переносы

\begin{document} % Конец преамбулы, начало текста.


\section*{Решение квадратного уравнения}

\emph{Задача}: решить уравнение $2x^2 + 5x - 12 = 0$.

\emph{Решение}. Это квадратное уравнение, общий вид которого:
\[ ax^2 + bx + c = 0 \]

В нашем случае $a = 2$, $b = 5$, $c = -12$.

Сначала необходимо вычислить дискриминант уравнения:
\[
D = b^2 - 4ac = (5)^2 - 4 \cdot 2 \cdot (-12) = 121
\]

Так как дискриминант является положительным ($D > 0$), это уравнение имеет два корня, вычисляемые по формуле:
\[
x_{1,2} = \frac{-b \pm \sqrt{D}}{2a} = \frac{-5 \pm 11}{4}
\]

Таким образом, $x_1 = \frac{-16}{4} = -4, \quad x_2 = \frac{6}{4} = \frac{3}{2}$

\emph{Ответ}: $x_1 = -4$, $x_2 = \frac{3}{2}$.

\end{document} % Конец текста.

